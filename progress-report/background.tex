\section{Background}
\label{sec:background}

\subsection{Base project: vivado-risc-v}
This project uses Rocket Chip to generate RISC-V processors that can be run on various FPGAs. It provides enough documentation to build a core and flash it to the board but very little else. There is some documentation on adding a peripheral device and making it available in Linux but this did not work and further research into how to add to the device tree resulted in limited information. It provides an example bare-metal file that can be used as a base for new code but there was no documentation on it.

\newpage
\subsection{AXI protocol}
The AXI4 protocol is used to communicate with custom IP blocks \cite{axi}. There are 3 interfaces AXI4, AXI4-Lite and AXI4-Stream. AXI4 provides high-performance memory-mapped communication and AXI4-Stream allows for high-speed streaming of data. For this project, AXI4-Lite will be used as it is a simpler interface for low-throughput memory-mapped communication. This will limit the performance of the accelerator but will simplify development. The project will be designed in a modular way to allow the interface block to be changed without needing to modify the accelertor allowing the option to utilise the higher-performance AXI4 interface in the future.

\subsection{Hardware Acceleration}
A hardware accelerator is a block of hardware designed to fully exploit the parallelism available in an algorithm. An example would be matrix multiplication. The naive method takes $\mathcal{O}(n^{3})$ time which can be optimised by more complicated algorithms to $\mathcal{O}(n^{2.3728596})$ \citep{alman2020refined}. Looking at the naive algorithm, it shows that while many operations are needed all the multiplications can be done in parallel and then added to generate the result. A hardware device would be able to do this allowing for it to be completed in constant time but would face issues with scaling.

Another possible application is in image processing with local filters \cite{image_process}. A pixle value is modified based on surrounding pixels in a set window using a matrix of values that scale each pixel in the window and sum to get the resulting pixel value. This can be used for various operations such as a Gaussian blur or edge detection using a Sobel filter. This process can be performed in parallel to calculate all the multiplications in a single window simultaneously or to calculate multiple windows simultaneously. The data required by a window is not all sequential so a cache is needed to hold the unused values from the input data stream until they will be needed. This can be optimised at a hardware level or by re-ordering the input data.